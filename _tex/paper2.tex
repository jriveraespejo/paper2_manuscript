% Options for packages loaded elsewhere
\PassOptionsToPackage{unicode}{hyperref}
\PassOptionsToPackage{hyphens}{url}
\PassOptionsToPackage{dvipsnames,svgnames,x11names}{xcolor}
%
\documentclass[
  authoryear,
  review,
  1p]{elsarticle}

\usepackage{amsmath,amssymb}
\usepackage{iftex}
\ifPDFTeX
  \usepackage[T1]{fontenc}
  \usepackage[utf8]{inputenc}
  \usepackage{textcomp} % provide euro and other symbols
\else % if luatex or xetex
  \usepackage{unicode-math}
  \defaultfontfeatures{Scale=MatchLowercase}
  \defaultfontfeatures[\rmfamily]{Ligatures=TeX,Scale=1}
\fi
\usepackage{lmodern}
\ifPDFTeX\else  
    % xetex/luatex font selection
\fi
% Use upquote if available, for straight quotes in verbatim environments
\IfFileExists{upquote.sty}{\usepackage{upquote}}{}
\IfFileExists{microtype.sty}{% use microtype if available
  \usepackage[]{microtype}
  \UseMicrotypeSet[protrusion]{basicmath} % disable protrusion for tt fonts
}{}
\makeatletter
\@ifundefined{KOMAClassName}{% if non-KOMA class
  \IfFileExists{parskip.sty}{%
    \usepackage{parskip}
  }{% else
    \setlength{\parindent}{0pt}
    \setlength{\parskip}{6pt plus 2pt minus 1pt}}
}{% if KOMA class
  \KOMAoptions{parskip=half}}
\makeatother
\usepackage{xcolor}
\setlength{\emergencystretch}{3em} % prevent overfull lines
\setcounter{secnumdepth}{5}
% Make \paragraph and \subparagraph free-standing
\makeatletter
\ifx\paragraph\undefined\else
  \let\oldparagraph\paragraph
  \renewcommand{\paragraph}{
    \@ifstar
      \xxxParagraphStar
      \xxxParagraphNoStar
  }
  \newcommand{\xxxParagraphStar}[1]{\oldparagraph*{#1}\mbox{}}
  \newcommand{\xxxParagraphNoStar}[1]{\oldparagraph{#1}\mbox{}}
\fi
\ifx\subparagraph\undefined\else
  \let\oldsubparagraph\subparagraph
  \renewcommand{\subparagraph}{
    \@ifstar
      \xxxSubParagraphStar
      \xxxSubParagraphNoStar
  }
  \newcommand{\xxxSubParagraphStar}[1]{\oldsubparagraph*{#1}\mbox{}}
  \newcommand{\xxxSubParagraphNoStar}[1]{\oldsubparagraph{#1}\mbox{}}
\fi
\makeatother


\providecommand{\tightlist}{%
  \setlength{\itemsep}{0pt}\setlength{\parskip}{0pt}}\usepackage{longtable,booktabs,array}
\usepackage{calc} % for calculating minipage widths
% Correct order of tables after \paragraph or \subparagraph
\usepackage{etoolbox}
\makeatletter
\patchcmd\longtable{\par}{\if@noskipsec\mbox{}\fi\par}{}{}
\makeatother
% Allow footnotes in longtable head/foot
\IfFileExists{footnotehyper.sty}{\usepackage{footnotehyper}}{\usepackage{footnote}}
\makesavenoteenv{longtable}
\usepackage{graphicx}
\makeatletter
\newsavebox\pandoc@box
\newcommand*\pandocbounded[1]{% scales image to fit in text height/width
  \sbox\pandoc@box{#1}%
  \Gscale@div\@tempa{\textheight}{\dimexpr\ht\pandoc@box+\dp\pandoc@box\relax}%
  \Gscale@div\@tempb{\linewidth}{\wd\pandoc@box}%
  \ifdim\@tempb\p@<\@tempa\p@\let\@tempa\@tempb\fi% select the smaller of both
  \ifdim\@tempa\p@<\p@\scalebox{\@tempa}{\usebox\pandoc@box}%
  \else\usebox{\pandoc@box}%
  \fi%
}
% Set default figure placement to htbp
\def\fps@figure{htbp}
\makeatother

\makeatletter
\@ifpackageloaded{caption}{}{\usepackage{caption}}
\AtBeginDocument{%
\ifdefined\contentsname
  \renewcommand*\contentsname{Table of contents}
\else
  \newcommand\contentsname{Table of contents}
\fi
\ifdefined\listfigurename
  \renewcommand*\listfigurename{List of Figures}
\else
  \newcommand\listfigurename{List of Figures}
\fi
\ifdefined\listtablename
  \renewcommand*\listtablename{List of Tables}
\else
  \newcommand\listtablename{List of Tables}
\fi
\ifdefined\figurename
  \renewcommand*\figurename{Figure}
\else
  \newcommand\figurename{Figure}
\fi
\ifdefined\tablename
  \renewcommand*\tablename{Table}
\else
  \newcommand\tablename{Table}
\fi
}
\@ifpackageloaded{float}{}{\usepackage{float}}
\floatstyle{ruled}
\@ifundefined{c@chapter}{\newfloat{codelisting}{h}{lop}}{\newfloat{codelisting}{h}{lop}[chapter]}
\floatname{codelisting}{Listing}
\newcommand*\listoflistings{\listof{codelisting}{List of Listings}}
\makeatother
\makeatletter
\makeatother
\makeatletter
\@ifpackageloaded{caption}{}{\usepackage{caption}}
\@ifpackageloaded{subcaption}{}{\usepackage{subcaption}}
\makeatother
\journal{Psychometrika}

\usepackage[]{natbib}
\bibliographystyle{elsarticle-harv}
\usepackage{bookmark}

\IfFileExists{xurl.sty}{\usepackage{xurl}}{} % add URL line breaks if available
\urlstyle{same} % disable monospaced font for URLs
\hypersetup{
  pdftitle={Let's talk about Thurstone \& Co.: An information-theoretical model for comparative judgments, and its statistical translation},
  pdfauthor={Jose Manuel Rivera Espejo; Tine van van Daal; Sven De De Maeyer; Steven Gillis},
  pdfkeywords={causal inference, directed acyclic graphs, structural
causal models, bayesian statistical methods, thurstonian
model, comparative judgement, probability, statistical modeling},
  colorlinks=true,
  linkcolor={blue},
  filecolor={Maroon},
  citecolor={Blue},
  urlcolor={Blue},
  pdfcreator={LaTeX via pandoc}}


\setlength{\parindent}{6pt}
\begin{document}

\begin{frontmatter}
\title{Let's talk about Thurstone \& Co.: An information-theoretical
model for comparative judgments, and its statistical translation}
\author[1]{Jose Manuel Rivera Espejo%
\corref{cor1}%
}
 \ead{JoseManuel.RiveraEspejo@uantwerpen.be} 
\author[1]{Tine van Daal%
%
}
 \ead{tine.vandaal@uantwerpen.be} 
\author[1]{Sven De Maeyer%
%
}
 \ead{sven.demaeyer@uantwerpen.be} 
\author[2]{Steven Gillis%
%
}
 \ead{steven.gillis@uantwerpen.be} 

\affiliation[1]{organization={University of Antwerp, Training and
education sciences},,postcodesep={}}
\affiliation[2]{organization={University of
Antwerp, Linguistics},,postcodesep={}}

\cortext[cor1]{Corresponding author}




        
\begin{abstract}
This study revisits Thurstone's law of comparative judgments (CJ) by
addressing two key limitations in traditional approaches. Firstly, it
addresses the overreliance on the assumptions of Thurstone's Case V in
the statistical analysis of CJ data. Secondly, it addresses the apparent
disconnect between CJ's approach to trait measurement and hypothesis
testing. We put forward a systematic approach based on causal analysis
and Bayesian statistical methods, which results in a model that
facilitates a more comprehensive understanding of the factors
influencing CJ experiments while offering a robust statistical
translation. The new model accommodates unequal dispersions and
correlations between stimuli, enhancing the reliability and validity of
CJ's trait estimation, thereby ensuring the accurate measurement and
interpretation of comparative data. The paper highlights the relevance
of this updated framework for modern empirical research, particularly in
education and social sciences. This contribution advances current
research methodologies, providing a robust foundation for future
applications in diverse fields.
\end{abstract}





\begin{keyword}
    causal inference \sep directed acyclic graphs \sep structural causal
models \sep bayesian statistical methods \sep thurstonian
model \sep comparative judgement \sep probability \sep 
    statistical modeling
\end{keyword}
\end{frontmatter}
    

\section{Introduction}\label{sec-introduction}

In \emph{comparative judgment} (CJ) studies, judges assess a specific
trait or attribute across different stimuli by performing pairwise
comparisons \citep{Thurstone_1927a, Thurstone_1927b}. Each comparison
produces a dichotomous outcome, indicating which stimulus is perceived
to have a higher trait level. For example, when assessing writing
quality, judges compare pairs of written texts (the stimuli) to
determine the relative writing quality each text exhibit (the trait)
\citep{Laming_2004, Pollitt_2012b, Whitehouse_2012, vanDaal_et_al_2016, Lesterhuis_2018_thesis, Coertjens_et_al_2017, Goossens_et_al_2018, Bouwer_et_al_2023}.

Numerous studies have documented the effectiveness of CJ in assessing
traits and competencies over the past decade. These studies have
highlighted three aspects of the method's effectiveness: its
reliability, validity, and practical applicability. Research on
reliability suggests that CJ requires a relatively modest number of
pairwise comparisons \citep{Verhavert_et_al_2019, Crompvoets_et_al_2022}
to generate trait scores that are as precise and consistent as those
generated by other assessment methods
\citep{Coertjens_et_al_2017, Goossens_et_al_2018, Bouwer_et_al_2023}. In
addition, the evidence suggests that the reliability and time efficiency
of CJ are comparable, if not superior, to those of other assessment
methods when employing adaptive comparison algorithms
\citep{Pollitt_2012b, Verhavert_et_al_2022, Mikhailiuk_et_al_2021}.
Meanwhile, research on the validity of CJ scores indicates their
capacity to represent the traits under measurement accurately
\citep{Whitehouse_2012, vanDaal_et_al_2016, Lesterhuis_2018_thesis, Bartholomew_et_al_2018, Bouwer_et_al_2023}.
Moreover, research on CJ's practical applicability highlights its
versatility across both educational and non-educational contexts
\citep{Kimbell_2012, Jones_et_al_2015, Bartholomew_et_al_2018, Jones_et_al_2019, Marshall_et_al_2020, Bartholomew_et_al_2020, Boonen_et_al_2020}.

Nevertheless, despite the increasing number of CJ studies, the
prevalence of unsystematic and fragmented research approaches has left
several critical issues unaddressed. The present study primarily focuses
on two issues: the overreliance on Thurstone's Case V assumptions in the
statistical analysis of CJ data and the apparent disconnect between CJ's
approach to trait measurement and hypothesis testing. The following
sections begin with a brief overview of Thurstone's theory followed by a
detailed examination of these issues. Subsequently, the study introduces
a theoretical model for CJ that builds upon Thurstone's theory,
alongside its statistical translation, designed to address the two
concerns simultaneously.

\section{Thurstone's theory}\label{sec-thurstone_theory}

In its most general form, Thurstone's theory addresses pairwise
comparisons wherein a single judge evaluates multiple stimuli
\citep[pp.~267]{Thurstone_1927b}. The theory posits that two key factors
determine the dichotomous outcome of these comparisons: the discriminal
process of each stimulus and their discriminal difference. The
\emph{discriminal process} captures the psychological impact each
stimulus exerts on the judge or, more simply, his perception of the
stimulus trait. The theory assumes that the discriminal process for any
given stimulus forms a Normal distribution along the trait continuum
\citep[pp.~266]{Thurstone_1927b}. The mode (mean) of this distribution,
known as the \emph{modal discriminal process}, indicates the stimulus
position on this continuum, while its dispersion, referred to as the
\emph{discriminal dispersion}, reflects variability in the perceived
trait of the stimulus.

Figure~\ref{fig-discriminal_process} illustrates hypothetical
discriminal processes along a quality trait continuum for two written
texts. The figure indicates that the modal discriminal process for Text
B is positioned further along the continuum than that of Text A
\((T_{B} > T_{A})\), suggesting that Text B exhibits higher quality.
Additionally, the figure highlights that Text B has a broader
distribution compared to Text A, which arises from its larger
discriminal dispersion \((\sigma_{B} > \sigma_{A})\).

\begin{figure}

\begin{minipage}{0.50\linewidth}

\centering{

\includegraphics[width=1\linewidth,height=\textheight,keepaspectratio]{./images/png/discriminal_process.png}

}

\subcaption{\label{fig-discriminal_process}Discriminal processes}

\end{minipage}%
%
\begin{minipage}{0.50\linewidth}

\centering{

\includegraphics[width=1\linewidth,height=\textheight,keepaspectratio]{./images/png/discriminal_difference.png}

}

\subcaption{\label{fig-discriminal_difference}Discriminal difference}

\end{minipage}%

\caption{\label{fig-thurstone_theory}Hypothetical discriminal processes
and discriminant difference along a quality trait continuum for two
written texts.}

\end{figure}%

However, given that the individual discriminal processes of the stimuli
are not directly observable, the theory introduces the \emph{law of
comparative judgment}. This law posits that in pairwise comparisons, a
judge perceives the stimulus with a discriminal process positioned
further along the trait continuum as possessing more of the trait
\citep[pp.~251]{Bramley_2008}. This suggests that the relative distance
between stimuli, rather than their absolute positions on the continuum,
likely defines the outcome of pairwise comparisons. Indeed, the theory
assumes that the difference between the underlying discriminal processes
of the stimuli, referred to as the \emph{discriminal difference},
determines the observed dichotomous outcome. Furthermore, the theory
assumes that because the individual discriminal processes form a Normal
distribution on the continuum, the discriminal difference will also
conform to a Normal distribution \citep{Andrich_1978}. In this
distribution, the mode (mean) represents the relative separation between
the stimuli, and its dispersion indicates the variability of that
separation.

Figure~\ref{fig-discriminal_difference} illustrates the distribution of
the discriminal difference for the hypothetical texts depicted in
Figure~\ref{fig-discriminal_process}. The figure indicates that the
judge perceives Text B as having significantly higher quality than Text
A. This conclusion is supported by two key observations: the positive
difference between their modal discriminal processes
\((T_{B} - T_{A} > 0)\) and the probability area where the discriminal
difference distinctly favors Text B over Text A, represented by the
shaded gray area denoted as \(P(B > A)\). As a result, the dichotomous
outcome of this comparison is more likely to favor Text B over Text A.

\section{The two critical issues in CJ
literature}\label{sec-theory-issues}

This section examines the two critical issues in the CJ literature that
serve as the primary focus of the present study. The first is related to
the overreliance on Thurstone's Case V assumptions in the statistical
analysis of CJ data. The second concern with the apparent disconnect
between CJ's approach to trait measurement and hypothesis testing.

\subsection{The Case V and the statistical analysis of CJ
data}\label{sec-theory-issue1}

Thurstone noted from the outset that the general form of the theory, as
outlined in Section~\ref{sec-thurstone_theory}, gave rise to a problem
of trait scaling. The model required estimating more ``unknown''
parameters than the available pairwise comparisons
\citep[pp.~267]{Thurstone_1927b}. To address this issue and facilitate
the practical implementation of the theory, he developed five cases
derived from this general form, each case progressively incorporated
additional simplifying assumptions into the model.

In Case I, Thurstone postulated that pairs of stimuli would maintain a
constant correlation across all comparisons. In Case II, he allowed
multiple judges to undertake comparisons instead of confining
evaluations to a single judge. In Case III, he posited that there was no
correlation between stimuli. In Case IV, he assumed that the stimuli
exhibited similar dispersions. Finally, in Case V, he replaced this
assumption with the condition that stimuli had equal discriminal
dispersions. Table~\ref{tbl-thurstone_cases} summarizes the assumptions
of the general form and the five cases. For a detailed discussion of
these cases and their progression, refer to \citet{Thurstone_1927b} and
\citet[pp.~248--253]{Bramley_2008}.

\begin{table}

\caption{\label{tbl-thurstone_cases}Thurstones cases and their
asumptions}

\centering{

\includegraphics[width=1\linewidth,height=\textheight,keepaspectratio]{./images/png/thurstone_cases.png}

}

\end{table}%

Notably, despite relying on the most extensive set of simplifying
assumptions
\citetext{\citealp[pp.~253]{Bramley_2008}; \citealp[pp.~677]{Kelly_et_al_2022}},
Case V remains the most widely used case in the CJ literature. This
popularity stems mainly from its simplified statistical representation
in the Bradley-Terry-Luce (BTL) model
\citep{Bradley_et_al_1952, Luce_1959}. The BTL model mirrors the
assumptions of Case V, with one notable distinction: whereas Case V
assumes a Normal distribution for the stimuli's discriminal processes,
the BTL model uses the more mathematically tractable Logistic
distribution \citep[pp.~254]{Andrich_1978, Bramley_2008} (see
Table~\ref{tbl-thurstone_cases}). This substitution has little impact on
the model's estimation or interpretation, as the Normal and Logistic
distributions exhibit analogous statistical properties, differing only
by a scaling factor of approximately \(1.7\)
\citep[pp.~16]{vanderLinden_et_al_2017_I}.

However, Thurstone originally developed Case V to provide a ``rather
coarse scaling'' of traits \citep[pp.~269]{Thurstone_1927b},
prioritizing statistical simplicity over precision in trait measurement
\citep[pp.~677]{Kelly_et_al_2022}. He explicitly warned against its
untested application, stating that its use ``should not be made without
(an) experimental test'' \citep[pp.~270]{Thurstone_1927b}. Furthermore,
he acknowledged that some assumptions could prove problematic when
researchers assess complex traits or heterogeneous stimuli
\citep[pp.~376]{Thurstone_1927a}. Consequently, given that modern CJ
applications frequently involve such traits and stimuli, two main
assumptions of Case V and, by extension, of the BTL model may not
consistently hold in theory or practice, namely the assumption of equal
dispersion and zero correlation between stimuli.

\subsubsection{The assumption of equal dispersions between
stimuli}\label{sec-theory-issue1a}

According to the theory, discrepancies in the discriminal dispersions of
stimuli shape the distribution of the discriminal difference, exerting a
direct influence on the outcome of pairwise comparisons.
Figure~\ref{fig-dispersion} presents a thought experiment to illustrate
this idea. In this experiment, a researcher can observe the discriminal
processes for the texts depicted in
Figure~\ref{fig-discriminal_process}. Furthermore, the figure assumes
that the discriminal dispersion for Text A remains constant and that the
texts are uncorrelated \((\rho=0)\). The figure reveals that an increase
in the uncertainty associated with the perception of Text B in
comparison to Text A, \((\sigma_{B}-\sigma_{A})\), broadens the
distribution of their discriminal difference. This broadening affects
the probability area where the discriminal difference distinctly favors
Text B over Text A, expressed as \(P(B > A)\), ultimately influencing
the comparison outcome. Additionally, the figure reveals that when the
discriminal dispersions of the texts are equal
\((\sigma_{B}-\sigma_{A}=0)\), the discriminal difference is more likely
to favor Text B over Text A (shaded gray area), compared to situations
where their dispersions differ.

\begin{figure}

\begin{minipage}{0.50\linewidth}

\centering{

\includegraphics[width=1\linewidth,height=\textheight,keepaspectratio]{./images/png/dispersion.png}

}

\subcaption{\label{fig-dispersion}Discrepancies in the dispersions of
stimuli}

\end{minipage}%
%
\begin{minipage}{0.50\linewidth}

\centering{

\includegraphics[width=1\linewidth,height=\textheight,keepaspectratio]{./images/png/correlation.png}

}

\subcaption{\label{fig-correlation}Correlation between stimuli}

\end{minipage}%

\caption{\label{fig-caseV_issues}The effect of dispersion discrepancies
and stimulus correlation on the distribution of the discriminal
difference.}

\end{figure}%

In experimental practice, however, this process occurs in reverse.
Researchers first observe the comparison outcome and then use the BTL
model to infer the discriminal difference between the stimuli and their
respective discriminal processes \citep[pp.~373]{Thurstone_1927a}.
Therefore, the outcome's ability to reflect the ``true'' differences
between stimuli largely depends on the validity of the model's
assumptions \citep[pp.~150]{Kohler_et_al_2019}, particularly the
assumption of equal dispersions. For instance, when researchers observe
a sample of outcomes favoring Text B over Text A and correctly assume
equal dispersions between the texts, the BTL model estimates a
discriminal difference distribution that accurately represents the
``true'' discriminal difference of the texts. This scenario is
illustrated in Figure~\ref{fig-dispersion}, where the model's
discriminal difference distribution aligns with the ``true''
distribution, represented by the thick continuous line corresponding to
\(\sigma_{B}-\sigma_{A}=0\). The accuracy of these discriminal
difference ensures reliable estimates for the texts' discriminal
processes {(citation needed?)}.

However, Thurstone argued that the assumption of equal dispersions may
not be applicable when researchers assess complex traits or
heterogeneous stimuli \citep[pp.~376]{Thurstone_1927a}, as these traits
and stimuli can introduce judgment discrepancies due to their unique
features
\citep{vanDaal_et_al_2016, Lesterhuis_2018, Chambers_et_al_2022}.
Indeed, evidence of this violation may already be present in the CJ
literature in the form of misfit statistics, which measure judgment
discrepancies associated with specific stimuli
\citetext{\citealp[pp.~12]{Pollitt_2004}; \citealp[pp.~20]{Goossens_et_al_2018}}.
For example, labeling texts as ``misfits'' indicates that comparisons
involving these texts result in more judgment discrepancies than those
involving other texts
\citep{Pollitt_2012a, Pollitt_2012b, vanDaal_et_al_2016, Goossens_et_al_2018}.
These discrepancies, in turn, suggest that the discriminal differences
for ``misfit'' texts have broader distributions, indicating that their
discriminal processes may also exhibit more variation than that of other
texts. A similar line of reasoning applies to the concept of ``misfit''
judges, whose evaluations deviate substantially from the shared
consensus due to the unique characteristics of the stimuli or the judges
themselves. These ``misfit'' judges and their associated deviations can
give rise to additional statistical and measurement issues, which we
discuss in more detail in Section~\ref{sec-theory-issue1b}.

Thus, model misspecification, in the form of an erroneous assumption of
equal dispersions between stimuli, can give rise to significant
statistical and measurement issues. For instance, the model may
overestimate the degree to which the outcome accurately reflects the
``true'' discriminal differences between stimuli. This overestimation
can result in researchers drawing spurious conclusions about these
differences \citep[pp.~370]{McElreath_2020} and, by extension, about the
underlying discriminal processes of stimuli. Figure~\ref{fig-dispersion}
also illustrates this issue when the model's discriminal difference
distribution aligns with the thick continuous line for
\(\sigma_{B}-\sigma_{A}=0\), while the ``true'' discriminal difference
follows any discontinuous line where \(\sigma_{B}-\sigma_{A} \neq 0\).
Additionally, if researchers recognize that misfit statistics highlight
these critical differences in dispersions, the conventional CJ practice
of excluding stimuli based on these statistics
\citep{Pollitt_2012a, Pollitt_2012b, vanDaal_et_al_2016, Goossens_et_al_2018}
can unintentionally discard valuable information. Such exclusions can
introduce bias into trait estimates
\citep[chap.~12]{Zimmerman_1994, McElreath_2020}. The direction and
magnitude of these biases are often unpredictable, as they depend on
which stimuli are excluded from the analysis.

\subsubsection{The assumption of zero correlation between
stimuli}\label{sec-theory-issue1b}

The correlation, represented by the symbol \(\rho\), measures how much a
judge's perception of a specific trait in one stimulus depends on their
perception of the same trait in another. As with the discriminal
dispersions, this correlation shapes the distribution of the discriminal
difference, directly impacting the outcomes of pairwise comparisons.
Figure~\ref{fig-correlation} presents a similar thought experiment as in
Section~\ref{sec-theory-issue1a} to illustrate this idea. The
illustration now assumes that the discriminal dispersions for both texts
remain constant. The figure reveals that as the correlation between the
texts increases, the distribution of their discriminal difference
becomes narrower. This narrowing affects the area under the curve where
the discriminal difference distinctly favors Text B over Text A, denoted
as \(P(B > A)\), thus influencing the comparison outcome. Furthermore,
the figure shows that when two texts are independent or uncorrelated
\((\rho=0)\), their discriminal difference is less likely to favor Text
B over Text A (shaded gray area) compared to scenarios where the texts
are highly correlated.

Off course, in experimental practice, researchers approach this process
in reverse. They begin by observing the sample of outcomes favoring Text
B over Text A and then use the BTL model to estimate the discriminal
difference and the discriminal processes of the stimuli. Given that the
BTL model assumes independent discriminal processes across comparisons,
if this assumption holds, then the model estimates a discriminal
difference distribution that accurately reflects the ``true''
discriminal difference of the texts. This scenario is also illustrated
in Figure~\ref{fig-correlation} when the discriminal difference
distribution of the model aligns with the ``true'' distribution,
represented by the thick continuous line corresponding to \(\rho=0\).
Once more, the estimation accuracy of the discriminal difference ensures
reliable estimates for the discriminal processes of the texts {(citation
needed?)}.

Notably, Thurstone attributed the independence of stimuli to the
cancellation of potential judges' biases. He argued that this
cancellation resulted from two opposing and equally weighted effects
occurring during pairwise comparisons \citep[pp.~268]{Thurstone_1927b}.
\citet{Andrich_1978} provided a mathematical demonstration of this
cancellation using the BTL model under the assumption of discriminal
processes with additive biases. However, it is easy to imagine at least
two scenarios in which the zero correlation assumption is almost
certainly invalid: when the pairwise comparison involves
multidimensional, complex traits with heterogeneous stimuli and when an
additional hierarchical structure is relevant to the stimuli.

In the first scenario, the intricate aspects of multidimensional,
complex traits may introduce dependencies between the stimuli due to
certain judges' biases that resist cancellation. Research on text
quality suggests that when judges evaluate these traits, they often rely
on various intricate characteristics of the stimuli to form their
judgments
\citep{vanDaal_et_al_2016, Lesterhuis_2018, Chambers_et_al_2022}. These
additional relevant characteristics, which are unlikely to be equally
weighted or opposing, can exert an uneven influence on judges'
perceptions, creating biases in their judgments and, ultimately,
introducing dependencies between stimuli
\citep[pp.~346]{vanderLinden_et_al_2017_II}. For example, this could
occur when a judge assessing the argumentative quality of a text places
more weight on its grammatical accuracy than other judges, thereby
favoring texts with fewer errors but weaker arguments. While direct
evidence for this particular scenario is lacking, studies such as
\citet{Pollitt_et_al_2003} demonstrate the presence of such biases,
supporting the notion that the factors influencing pairwise comparisons
may not always cancel out.

In the second scenario, the shared context or inherent connections
created by additional hierarchical structures may further introduce
dependencies between stimuli, a statistical phenomenon commonly known as
clustering \citep{Everitt_et_al_2010}. Despite the CJ literature
acknowledging the existence of such hierarchical structures, the
statistical handling of this additional source of dependence between
stimuli has been inadequate. For instance, when CJ data incorporates
multiple samples of stimuli from the same individuals, researchers
frequently rely on (average) estimated BTL scores to conduct subsequent
analyses and tests at the individual hierarchical level
\citep{Bramley_et_al_2019, Boonen_et_al_2020, Bouwer_et_al_2023, vanDaal_et_al_2017, Jones_et_al_2019, Gijsen_et_al_2021}.
However, this approach can introduce additional statistical and
measurement issues, which we discuss in greater detail in
Section~\ref{sec-theory-issue2}.

In any case, similar to Section~\ref{sec-theory-issue1a}, model
misspecification due to an erroneous assumption of zero correlation
between stimuli can lead to significant statistical and measurement
issues. For instance, the model may over- or underestimate how
accurately the outcome reflects the ``true'' discriminal differences
between stimuli. Such inaccuracies can result in spurious inferences
about these differences and, by extension, about the stimuli's
discriminal processes. This scenario is also illustrated by
Figure~\ref{fig-correlation}, when the model's discriminal difference
distribution aligns with the thick continuous line for \(\rho=0\), while
the ``true'' discriminal difference follows any discontinuous line where
\(\rho \neq 0\).

The misspecification may arise from neglecting additional relevant
traits, excluding judges based on misfit statistics, or ignoring
hierarchical (grouping) structures. Neglecting relevant traits, such as
judges' biases, can cause dimensional mismatches in the BTL model,
artificially inflating the trait's reliability
\citep[pp.~341]{Hoyle_et_al_2023} or, worse, introducing bias into the
trait's estimates \citep{Ackerman_1989}. Excluding judges based on
misfit statistics risks discarding valuable information, which may
further bias the trait's estimates
\citep[chap.~12]{Zimmerman_1994, McElreath_2020}. Finally, ignoring
hierarchical structures may reduce the precision of model parameter
estimates, potentially amplifying the overestimation of the trait's
reliability \citep[pp.~482]{Hoyle_et_al_2023}.

\subsection{The disconnect between trait measurement and hypothesis
testing}\label{sec-theory-issue2}

Building on the previous section, it is clear that, despite its
limitations, the BTL model is commonly used as a measurement model in CJ
assessments. A measurement model specifies how manifest variables
contribute to the estimation of latent variables
\citep{Everitt_et_al_2010}. For example, when evaluating writing
quality, researchers use the BTL model to process the dichotomous
outcomes resulting from the pairwise comparisons (the manifest
variables) to estimate scores that reflect the underlying level of
writing quality (the latent variable)
\citep{Laming_2004, Pollitt_2012b, Whitehouse_2012, vanDaal_et_al_2016, Lesterhuis_2018_thesis, Coertjens_et_al_2017, Goossens_et_al_2018, Bouwer_et_al_2023}.

Researchers then typically use these estimated BTL scores, or their
transformations, to conduct additional analyses or hypothesis tests. For
example, these scores have been used to identify `misfit' judges and
stimuli \citep{Pollitt_2012b, vanDaal_et_al_2016, Goossens_et_al_2018},
detect biases in judges' ratings
\citep{Pollitt_et_al_2003, Pollitt_2012b}, calculate correlations with
other assessment methods \citep{Goossens_et_al_2018, Bouwer_et_al_2023},
or test hypotheses related to the underlying trait of interest
\citep{Bramley_et_al_2019, Boonen_et_al_2020, Bouwer_et_al_2023, vanDaal_et_al_2017, Jones_et_al_2019, Gijsen_et_al_2021}.

However, the statistical literature advises caution when using estimated
scores for additional analyses and tests. A key consideration is that
BTL scores are parameter estimates that inherently carry uncertainty.
Ignoring this uncertainty can bias the analysis and reduce the precision
of hypothesis tests. Notably, the direction and magnitude of such biases
are often unpredictable. Results may be attenuated, exaggerated, or
remain unaffected depending on the degree of uncertainty in the scores
and the actual effects being tested
\citetext{\citealp[pp.~25]{Kline_et_al_2023}; \citealp[pp.~137]{Hoyle_et_al_2023}}.
Finally, the reduced precision in hypothesis tests diminishes their
statistical power, increasing the likelihood of committing type-I or
type-II errors \citep{McElreath_2020}.

In aggregate, researchers' inadequate handling of violations to the
assumptions of equal dispersion and zero correlation between stimuli,
coupled with the apparent disconnect between CJ's approach to trait
measurement and hypothesis testing, can potentially compromise the
reliability of the trait estimates and, by extension, their validity
\citep[pp.~2]{Perron_et_al_2015}. Consequently, adopting a more
systematic and integrated approach to handling these assumptions and
examining the factors influencing CJ experiments could offer several
statistical and measurement benefits, including the ability to address
these issues.

\section{Updating the theoretical and statistical
model}\label{sec-theory}

This section uses the structural approach to causal inference
\citep{Pearl_2009, Pearl_et_al_2016} to articulate a theoretical model
that captures the core principles of Thurstone's theory. The model also
incorporates various assessment design features relevant to CJ
experiments, such as the selection of judges, stimuli, and comparisons.
Finally, the section employs Bayesian statistical methods to transform
these theoretical and practical elements into a statistical model that
facilitates the analysis of pairwise comparison data. See
Section~\ref{sec-appendix} for an overview of the statistical and causal
inference concepts required for the development of this section.

\subsection{The theoretical model}\label{sec-theory-theoretical}

The theoretical model uses structural causal models (SCMs) and directed
acyclic graphs (DAGs)
\citep{Pearl_2009, Pearl_et_al_2016, Gross_et_al_2018, Neal_2020} to
formally and graphically represent the assumed causal structure of the
CJ system. First, \emph{the population model} is created to represent a
conceptual population of CJ experiments. The model then integrates
various assessment design features relevant to CJ experiments, leading
to the development of \emph{the sample-comparison model}.

\subsubsection{The population model}\label{sec-theory-theoretical_P}

Assuming population data or more commonly known as census data, we
\ldots{}

The (latent) discriminal difference of the stimuli directly determines
the (manifest) outcome of the pairwise comparisons

The (latent) ``perceived'' discriminal processes for the stimuli
directly determines their discriminal difference

The (latent) ``true'' discriminal processes for the stimuli and the
judges' biases directly determines their (latent) ``perceived''
discriminal processes

\begin{figure}

\centering{

\includegraphics[width=0.45\linewidth,height=\textheight,keepaspectratio]{./images/png/CJ_TM_04.png}

}

\caption{\label{fig-CJ_TM_04}}

\end{figure}%

without loosing generality, the (latent) ``perceived'' and ``true''
discriminal processes for the stimuli can be depicted in a vector for
each judge, as in

\begin{figure}

\centering{

\includegraphics[width=0.68\linewidth,height=\textheight,keepaspectratio]{./images/png/CJ_TM_10.png}

}

\caption{\label{fig-CJ_TM_10}}

\end{figure}%

\subsubsection{The sample-comparison
model}\label{sec-theory-theoretical_SC}

Considering the sampling mechanism

\begin{figure}

\centering{

\includegraphics[width=0.8\linewidth,height=\textheight,keepaspectratio]{./images/png/CJ_TM_12.png}

}

\caption{\label{fig-CJ_TM_12}}

\end{figure}%

Considering comparison mechanisms

\begin{figure}

\centering{

\includegraphics[width=0.9\linewidth,height=\textheight,keepaspectratio]{./images/png/CJ_TM_14.png}

}

\caption{\label{fig-CJ_TM_14}}

\end{figure}%

\subsection{From theory to statistics}\label{sec-theory-statistics}

\section{Discussion}\label{sec-discuss}

\subsection{Findings}\label{sec-discuss-finding}

\subsection{Limitations and further
research}\label{sec-discuss-limitations}

\section{Conclusion}\label{sec-conclusion}

\newpage{}

\section*{Declarations}\label{declarations}
\addcontentsline{toc}{section}{Declarations}

\textbf{Funding:} The project was founded through the Research Fund of
the University of Antwerp (BOF).

\textbf{Financial interests:} The authors have no relevant financial
interest to disclose.

\textbf{Non-financial interests:} The authors have no relevant
non-financial interest to disclose.

\textbf{Ethics approval:} The University of Antwerp Research Ethics
Committee has confirmed that no ethical approval is required.

\textbf{Consent to participate:} Not applicable

\textbf{Consent for publication:} All authors have read and agreed to
the published version of the manuscript.

\textbf{Availability of data and materials:} No data was utilized in
this study.

\textbf{Code availability:} All the code utilized in this research is
available in the digital document located at:
\url{https://jriveraespejo.github.io/paper2_manuscript/}.

\textbf{AI-assisted technologies in the writing process:} The authors
utilized a range of AI-based language tools throughout the preparation
of this work. They occasionally employed the tools to refine phrasing
and optimize wording, ensuring appropriate language use and enhancing
the manuscript's clarity and coherence. The authors take full
responsibility for the final content of the publication.

\textbf{CRediT authorship contribution statement:}
\emph{Conceptualization:} S.G., S.DM., T.vD., and J.M.R.E;
\emph{Methodology:} S.DM., T.vD., and J.M.R.E; \emph{Software:}
J.M.R.E.; \emph{Validation:} J.M.R.E.; \emph{Formal Analysis:} J.M.R.E.;
\emph{Investigation:} J.M.R.E; \emph{Resources:} S.G., S.DM., and T.vD.;
\emph{Data curation:} J.M.R.E.; \emph{Writing - original draft:}
J.M.R.E.; \emph{Writing - review and editing:} S.G., S.DM., and T.vD.;
\emph{Visualization:} J.M.R.E.; \emph{Supervision:} S.G. and S.DM.;
\emph{Project administration:} S.G. and S.DM.; \emph{Funding
acquisition:} S.G. and S.DM.

\newpage{}

\section{Appendix}\label{sec-appendix}

This section introduces fundamental statistical and causal inference
concepts necessary for understanding the core theoretical principles
described in Section~\ref{sec-theory}. It does not, however, offer a
comprehensive overview of causal inference methods. Readers seeking more
in-depth understanding may wish to explore introductory papers such as
\citet{Pearl_2010}, \citet{Rohrer_2018}, \citet{Pearl_2019}, and
\citet{Cinelli_et_al_2020}. They may also find it helpful to consult
introductory books like \citet{Pearl_et_al_2018}, \citet{Neal_2020}, and
\citet{McElreath_2020}. For more advanced study, readers may refer to
seminal intermediate papers such as \citet{Neyman_et_al_1923},
\citet{Rubin_1974}, \citet{Spirtes_et_al_1991}, and \citet{Sekhon_2009},
as well as books such as \citet{Pearl_2009}, \citet{Morgan_et_al_2014},
and \citet{Hernan_et_al_2020}.

\subsection{Empirical research and randomized
experiments}\label{sec-appendix-A}

Empirical research uses evidence from observation and experimentation to
address real-world challenges. In this context, researchers typically
formulate their research questions as \emph{estimands} or \emph{targets
of inference}, i.e., the specific quantities they seek to determine
\citep{Everitt_et_al_2010}. For instance, researchers might be
interested in answering the following question: ``To what extent do
different teaching methods \((T)\) influence students' ability to
produce high-quality written texts \((Y)\)?'' To investigate this,
researchers could randomly assign students to two groups, each exposed
to a different teaching method \((T_{i} = \{1,2\})\). Then, they would
perform pairwise comparisons, generating a dichotomous outcome
\((Y_{i} = \{0,1\})\) showing which student exhibits more of the
ability. In this scenario, the research question can be rephrased as the
estimand, ``\emph{On average}, is there a difference in the ability to
produce high-quality written texts between the two groups of students?''
and this estimand can be mathematically represented by the random
quantity \(E[Y_{i}| T_{i}=1] - E[Y_{i}| T_{i}=2]\), where \(E[\cdot]\)
denotes the expected value.

Researchers would then proceed to identify the estimands.
\emph{Identification} refers to the process of accurately computing an
estimand using an estimator. An \emph{estimator} is a method or function
that transforms data into an estimate \citep{Neal_2020}.
\emph{Estimates} are numerical values that approximate the estimand and
are derived through \emph{estimation}, which refers to the process of
integrating data with an estimator \citep{Everitt_et_al_2010}. The
Identification-Estimation flowchart \citep{McElreath_2020, Neal_2020} in
Figure~\ref{fig-IEflow} provides a visual representation of the process
of transitioning from estimands to estimates.

\begin{figure}

\centering{

\includegraphics[width=0.35\linewidth,height=\textheight,keepaspectratio]{images/png/IEflow.png}

}

\caption{\label{fig-IEflow}Identification-Estimation flowchart.
Extracted and slightly modified from \citet[pp.~32]{Neal_2020}}

\end{figure}%

While numerous methods can approximate an estimand, researchers
prioritize estimators with desirable properties that ensure the accuracy
of estimates. For instance, the Z-test is an estimator known for its
effectiveness in comparing groups' proportions, yielding accurate
estimates when the underlying assumptions of the statistic are met
\citep{Kanji_2006}. If this is the case, the Z-test is expressed as a
signal-to-noise statistic
\(Z = (\hat{p}_{1} - \hat{p}_{2})/ \hat{s}_{p}\). The signal is defined
as the difference between the groups' sample proportions,
\(\hat{p}_{1} = \sum_{i=1}^{n_{1}}{Y_{i}/n_{1}}\) and
\(\hat{p}_{2} = \sum_{i=1}^{n_{2}}{Y_{i}/n_{2}}\), analogous to
\(E[Y_{i}| T_{i}=1]\) and \(E[Y_{i}| T_{i}=2]\), respectively. The
noise, represented by \(\hat{s}_{p}\), is defined as the unpooled sample
variability observed between the two groups.

However, researchers often seek to uncover the mechanisms underlying
specific data and establish causal relationships rather than simply
estimate associations. In the example, researchers can interpret the
associational estimate represented by the Z-statistic as causal. This
interpretation relies on the data meeting the assumptions of the Z-test
and being collected through a randomized experiment.

Randomized experiments are widely recognized as the gold standard in
evidence-based science \citep{Hariton_et_al_2018, Hansson_2014}. This
recognition stems from their ability to enable researchers to interpret
associational estimates as causal. They achieve this by ensuring data,
and by extension an estimator, satisfies several key properties, such as
common support, no interference, and consistency
\citep{Morgan_et_al_2014, Neal_2020}. The most critical property,
however, is the elimination of confounding. \emph{Confounding} occurs
when an external variable \(X\) simultaneously influences the outcome
\(Y\) and the variable of interest \(T\), resulting in spurious
associations \citep{Everitt_et_al_2010}. Randomization addresses this
issue by decoupling the association between the intervention allocation
\(T\) from any other variable \(X\)
\citep{Morgan_et_al_2014, Neal_2020}.

Nevertheless, researchers often face constraints that limit their
ability to conduct randomized experiments. These constraints include
ethical concerns, such as the assignment of individuals to potentially
harmful interventions, and practical limitations, such as the
infeasibility of, for example, assigning individuals to genetic
modifications or physical impairments \citep{Neal_2020}. In these cases,
causal inference offers a valuable alternative for generating causal
estimates and understanding the mechanisms underlying specific data. In
addition, the framework can provide significant theoretical insights
that can enhance the design of experimental and observational studies
\citep{McElreath_2020}.

\subsection{Empirical research under causal
inference}\label{sec-appendix-B}

Unlike classical statistical modeling, which focuses primarily on
summarizing data and inferring associations, the \emph{causal inference}
framework is designed to identify causes and estimate their effects
using data \citep{Shaughnessy_et_al_2010, Neal_2020}. The framework uses
rigorous mathematical techniques to address the \emph{fundamental
problem of causality}
\citep{Pearl_2009, Pearl_et_al_2016, Morgan_et_al_2014}. This problem
revolves around the question, ``What would have happened `in the world'
under different circumstances?'' This question introduces the concept of
counterfactuals, which are instrumental in understanding and defining
causal effects.

\emph{Counterfactuals} are hypothetical scenarios that are
\emph{contrary to fact}, where alternative outcomes resulting from a
given cause are neither observed nor observable
\citep{Neal_2020, Counterfactual_2024}. The structural approach to
causal inference \citep{Pearl_2009, Pearl_et_al_2016} provides a formal
framework for defining counterfactuals. For instance, in the scenario
described in Section~\ref{sec-appendix-A}, the approach begins by
defining the \emph{individual causal effect} (ICE) as the difference
between each student's potential outcomes:
\(\tau_{i} = Y_{i}|do(T_{i}=1) - Y_{i}|do(T_{i}=2)\). Here,
\(do(T_{i}=t)\) represents the intervention operator,
\(Y_{i}|do(T_{i}=1)\) represents the potential outcome under
intervention \(T_{i}=1\), and \(Y_{i}|do(T_{i}=1)\) represents the
potential outcome under intervention \(T_{i}=2\). Note that if a student
is assigned to intervention \(T_{i}=1\), the potential outcome under
\(T_{i}=2\) becomes a counterfactual, as it is no longer observed nor
observable. To address the challenge of unobserved counterfactuals, the
approach extends the ICE to the \emph{average causal effect} (ACE):
\(\tau = E[\tau_{i}] = E[Y_{i}|do(T_{i}=1)]- E[Y_{i}|do(T_{i}=2)]\),
representing the average difference between observed potential outcomes
and their counterfactual counterparts. Notably, the approach extends the
ACEs for categorical and continuous variables \(T\), ensuring broad
applicability across different causal scenarios.

Even when data originates from an observational study, researchers can
identify the ACE from associational estimates using the structural
approach. They achieve this by performing statistical conditioning on a
\emph{sufficient adjustment set} of variables \(X\)
\citep{Pearl_2009, Pearl_et_al_2016, Morgan_et_al_2014}. This
\emph{sufficient} set (potentially empty) must block all non-causal
paths between \(T\) to \(Y\) without opening new ones, ensuring the ACE
estimator satisfies several key properties, including confounding
elimination. If such a set exists, then \(T\) and \(Y\) are
\emph{d-separated} by \(X\) \citep{Pearl_2009}, meaning researchers can
estimate the ACE from associational random quantities
\citep{Morgan_et_al_2014}. Naturally, the validity of claims about the
effects of \(T\) on \(Y\) now hinges on the assumption that \(X\) serves
as a sufficient adjustment set. However, as
\citet[pp.~150]{Kohler_et_al_2019} observed, drawing conclusions about
the real world from observed data inevitably requires assumptions. This
holds true for both observational and experimental data.

For instance, if researchers are unable to conduct the randomized
experiments described in Section~\ref{sec-appendix-A} and instead rely
on observational data, they can still estimate the ACE provided a
variable \(X\), such as the socio-economic status of the school, blocks
all non-causal paths between the teaching method \(T\) to the outcome
\(Y\) without opening any new ones. Under these conditions, the ACE can
be estimated from associational quantities as
\(\tau = E_{X}\left[ E[Y_{i}| T_{i}=1, X] - E[Y_{i}| T_{i}=2, X] \right]\),
where \(E_{X}[\cdot]\) represents the marginal expected value over \(X\)
\citep{Morgan_et_al_2014}.

\subsection{SCMs and DAGs}\label{sec-appendix-C}

\newcommand{\dsep}{\perp\!\!\!\perp}
\newcommand{\ndsep}{\not\!\perp\!\!\!\perp}

The structural approach to causal inference uses SCMs and DAGs to
formally and graphically represent the presumed causal structure
underlying the ACE
\citep{Pearl_2009, Pearl_et_al_2016, Gross_et_al_2018, Neal_2020}. In
essence, SCMs and DAGs act as \emph{conceptual models} that guide
researchers in determining which statistical models can yield valid
causal inferences, assuming the depicted causal structure of the models
is correct \citep{McElreath_2020}. Notably, every SCM has an associated
DAG \citep{Cinelli_et_al_2020}. Figure~\ref{fig-IEflow} provides a
visual representation of the role of theoretical models in the inference
process.

SCMs and DAGs offer two key advantages for modeling causal structures.
First, they enable the representation of causal relations in a
non-parametric and fully interactive manner. This feature allows for
feasible ACE identification strategies without specifying the data type
or the nature of the functional dependence among variables
\citep{Morgan_et_al_2014}. Second, regardless of complexity, they can
represent a wide range of causal structures using only five fundamental
building blocks \citep{Neal_2020, McElreath_2020}. This feature allows
researchers to decompose complex structures, facilitating their analysis
by focusing on the causal assumptions associated with each building
block \citep{McElreath_2020}.

Figures \ref{fig-dags_scms1}, \ref{fig-dags_scms2},
\ref{fig-dags_scms3}, \ref{fig-dags_scms4}, and \ref{fig-dags_scms5}
display these five fundamental building blocks. The left panels of the
figures show the formal mathematical models, represented by the SCMs,
defined in terms of a set of \emph{endogenous} variables
\(V=\{X,Z,Y\}\), a set of \emph{exogenous} variables
\(E=\{e_{X},e_{Z},e_{Y}\}\), and a set of functions
\(F=\{f_{X},f_{Z},f_{Y}\}\)
\citep{Pearl_2009, Cinelli_et_al_2020, Neal_2020}. Endogenous variables
are those whose causal mechanisms a researcher chooses to model
\citep{Neal_2020}. In contrast, exogenous variables represent
\emph{errors} or \emph{disturbances} arising from omitted factors that
the investigator chooses not to model explicitly
\citep[pp.~27,68]{Pearl_2009}. Lastly, the functions, referred to as
\emph{structural equations}, express the endogenous variables as
non-parametric functions of other variables. These functions use the
symbol `\(:=\)' to denote the asymmetrical causal dependence of the
variables and the symbol `\(\perp\!\!\!\perp\)' to represent
\emph{d-separation}, a concept akin to (conditional) independence.

The right panels of the figures display the complementary DAGs. A DAG
consists of nodes connected by edges, where nodes represent random
variables. The term \emph{directed} means the edges extend from one node
to another, with arrows indicating the direction of causal influence.
The term \emph{acyclic} signifies that the causal influences do not form
loops, ensuring the influences do not cycle back on themselves
\citep{McElreath_2020}. DAGs represent observed variables as solid black
circles, while they use open circles for unobserved (latent) variables
\citep{Morgan_et_al_2014}. Finally, the arrows in the graphs show the
expected direction of causal influences among these variables. While
DAGs often omit exogenous variables for simplicity (the \emph{standard}
representation), including these variables is advantageous (the
\emph{magnified} representation shown in the figures), as their
inclusion helps to identify potential issues related to conditioning and
confounding \citep{Cinelli_et_al_2020}.

\begin{figure}[H]

\begin{minipage}{0.50\linewidth}

\centering{

\[
\begin{aligned}
  X & := f_{X}(e_{X}) \\
  Y & := f_{Y}(e_{Y}) \\
  e_{X} & \perp\!\!\!\perp e_{Y}
\end{aligned}
\]

}

\subcaption{\label{fig-scm_bb1}SCM}

\end{minipage}%
%
\begin{minipage}{0.50\linewidth}

\centering{

\includegraphics[width=0.5\linewidth,height=\textheight,keepaspectratio]{images/png/mdag_bb1.png}

}

\subcaption{\label{fig-mdag_bb1}DAG}

\end{minipage}%
\newline
\begin{minipage}{0.50\linewidth}

\end{minipage}%

\caption{\label{fig-dags_scms1}Two unconnected nodes}

\end{figure}%

\begin{figure}[H]

\begin{minipage}{0.50\linewidth}

\centering{

\[
\begin{aligned}
  X & := f_{X}(e_{X}) \\
  Y & := f_{Y}(X,e_{Y}) \\
  e_{X} & \perp\!\!\!\perp e_{Y}
\end{aligned}
\]

}

\subcaption{\label{fig-scm_bb2}SCM}

\end{minipage}%
%
\begin{minipage}{0.50\linewidth}

\centering{

\includegraphics[width=0.5\linewidth,height=\textheight,keepaspectratio]{images/png/mdag_bb2.png}

}

\subcaption{\label{fig-mdag_bb2}DAG}

\end{minipage}%
\newline
\begin{minipage}{0.50\linewidth}

\end{minipage}%

\caption{\label{fig-dags_scms2}Two connected nodes or descendant}

\end{figure}%

\begin{figure}[H]

\begin{minipage}{0.50\linewidth}

\centering{

\[
\begin{aligned}
  X & := f_{X}(e_{X}) \\
  Z & := f_{Z}(X,e_{Z}) \\
  Y & := f_{Y}(Z,e_{Y}) \\
  e_{X} & \perp\!\!\!\perp e_{Y} \\
  e_{X} & \perp\!\!\!\perp e_{Z} \\
  e_{Z} & \perp\!\!\!\perp e_{Y}
\end{aligned}
\]

}

\subcaption{\label{fig-scm_bb3}SCM}

\end{minipage}%
%
\begin{minipage}{0.50\linewidth}

\centering{

\includegraphics[width=0.6\linewidth,height=\textheight,keepaspectratio]{images/png/mdag_bb3.png}

}

\subcaption{\label{fig-mdag_bb3}DAG}

\end{minipage}%
\newline
\begin{minipage}{0.50\linewidth}

\end{minipage}%

\caption{\label{fig-dags_scms3}Chain or mediator}

\end{figure}%

\begin{figure}[H]

\begin{minipage}{0.50\linewidth}

\centering{

\[
\begin{aligned}
  X & := f_{X}(Z,e_{X}) \\
  Z & := f_{Z}(e_{Z}) \\
  Y & := f_{Y}(Z,e_{Y}) \\
  e_{X} & \perp\!\!\!\perp e_{Y} \\
  e_{X} & \perp\!\!\!\perp e_{Z} \\
  e_{Z} & \perp\!\!\!\perp e_{Y}
\end{aligned}
\]

}

\subcaption{\label{fig-scm_bb4}SCM}

\end{minipage}%
%
\begin{minipage}{0.50\linewidth}

\centering{

\includegraphics[width=0.6\linewidth,height=\textheight,keepaspectratio]{images/png/mdag_bb4.png}

}

\subcaption{\label{fig-mdag_bb4}DAG}

\end{minipage}%
\newline
\begin{minipage}{0.50\linewidth}

\end{minipage}%

\caption{\label{fig-dags_scms4}Fork or confounder}

\end{figure}%

\begin{figure}[H]

\begin{minipage}{0.50\linewidth}

\centering{

\[
\begin{aligned}
  X & := f_{X}(e_{X}) \\
  Z & := f_{Z}(X,Y,e_{Z}) \\
  Y & := f_{Y}(e_{Y}) \\
  e_{X} & \perp\!\!\!\perp e_{Y} \\
  e_{X} & \perp\!\!\!\perp e_{Z} \\
  e_{Z} & \perp\!\!\!\perp e_{Y}
\end{aligned}
\]

}

\subcaption{\label{fig-scm_bb5}SCM}

\end{minipage}%
%
\begin{minipage}{0.50\linewidth}

\centering{

\includegraphics[width=0.6\linewidth,height=\textheight,keepaspectratio]{images/png/mdag_bb5.png}

}

\subcaption{\label{fig-mdag_bb5}DAG}

\end{minipage}%
\newline
\begin{minipage}{0.50\linewidth}

\end{minipage}%

\caption{\label{fig-dags_scms5}Collider or inmorality}

\end{figure}%

A careful examination of the figures highlights the assumptions
underlying these building blocks. Figures \ref{fig-scm_bb1} and
\ref{fig-mdag_bb1} depict two unconnected nodes, representing a scenario
where variables \(X\) and \(Y\) are not causally related. Figures
\ref{fig-scm_bb2} and \ref{fig-mdag_bb2} illustrate two connected nodes,
representing a scenario where a \emph{parent} node \(X\) exerts a causal
influence on a \emph{child} node \(Y\). Consequently, \(Y\) is
considered a \emph{descendant} of \(X\). Figures \ref{fig-scm_bb3} and
\ref{fig-mdag_bb3} depict a \emph{chain}, where \(X\) influences \(Z\),
and \(Z\) influences \(Y\). In this configuration, \(X\) is a parent
node of \(Z\), which is a parent node of \(Y\). This structure creates a
\emph{directed path} between \(X\) and \(Y\). Consequently, \(X\) is an
\emph{ancestor} of \(Y\), and \(Z\) fully \emph{mediates} the
relationship between the two. Figures \ref{fig-scm_bb4} and
\ref{fig-mdag_bb4} illustrate a \emph{fork}, where variables \(X\) and
\(Y\) are both influenced by \(Z\). Here, \(Z\) is a parent node that
\emph{confounds} the relationship between \(X\) and \(Y\). Finally,
Figures \ref{fig-scm_bb5} and \ref{fig-mdag_bb5} show a \emph{collider},
where variables \(X\) and \(Y\) are concurrent causes of \(Z\). In this
configuration, \(X\) and \(Y\) are not causally related to each other
but both influence \(Z\) (an \emph{inmorality}). Notably, in all SCMs,
the errors are assumed to be independent of each other and from all
other variables in the graph, as evidenced by the pairwise relations
\(e_{X} \perp\!\!\!\perp e_{Y}\), \(e_{X} \perp\!\!\!\perp e_{Z}\), and
\(e_{Z} \perp\!\!\!\perp e_{Y}\).

Researchers can use these building blocks to represent the scenario
outlined in Section~\ref{sec-appendix-B}. Figures \ref{fig-scm_example1}
and \ref{fig-mdag_example1} depict the plausible causal structure for
this example, assuming that the variable \(X\) (socio-economic status of
the school) acts as a confounder in the relationship between the
teaching method \(T\) and the outcome \(Y\). The figures show several
descendant, such as \(X \rightarrow T\), \(X \rightarrow Y\), and
\(T \rightarrow Y\), and also highlight multiple pairs of unconnected
nodes, evident from the relationships \(e_{T} \perp\!\!\!\perp e_{X}\),
\(e_{T} \perp\!\!\!\perp e_{Y}\), and \(e_{X} \perp\!\!\!\perp e_{Y}\).
Additional, the figures depict one fork, \(X \rightarrow \{T, Y\}\), and
two colliders: \(\{X, e_{T}\} \rightarrow T\) and
\(\{X, T, e_{Y}\} \rightarrow Y\).

\begin{figure}

\begin{minipage}{0.50\linewidth}

\centering{

\[
\begin{aligned}
  X & := f_{X}(e_{X}) \\
  T & := f_{Z}(X,e_{T}) \\
  Y & := f_{Y}(T,X,e_{Y}) \\
  e_{T} & \perp\!\!\!\perp e_{X} \\
  e_{T} & \perp\!\!\!\perp e_{Y} \\
  e_{X} & \perp\!\!\!\perp e_{Y}
\end{aligned}
\]

}

\subcaption{\label{fig-scm_example1}SCM}

\end{minipage}%
%
\begin{minipage}{0.50\linewidth}

\centering{

\includegraphics[width=0.6\linewidth,height=\textheight,keepaspectratio]{images/png/mdag_example1.png}

}

\subcaption{\label{fig-mdag_example1}DAG}

\end{minipage}%
\newline
\begin{minipage}{0.50\linewidth}

\end{minipage}%

\caption{\label{fig-example}Plausible causal structure the scenario
outlined in Section~\ref{sec-appendix-B}.}

\end{figure}%

\subsection{The flow of association and causation in
DAGs}\label{sec-appendix-D}

\begin{figure}

\centering{

\includegraphics[width=0.8\linewidth,height=\textheight,keepaspectratio]{images/png/ACflow.png}

}

\caption{\label{fig-ACflow}The flow of association and causation in
graphs. Extracted and slightly modified from \citet[pp.~31]{Neal_2020}}

\end{figure}%

\newpage{}

\section*{References}\label{references}
\addcontentsline{toc}{section}{References}

\renewcommand{\bibsection}{}
\bibliography{references.bib}





\end{document}
